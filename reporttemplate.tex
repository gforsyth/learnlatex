%%%%%%%%%%%%%%%%%%%%%%%%%%%%%%%%%%%%%%
% Comments are started with the % sign
%
%         File: reporttemplate.tex
% Date Created: 2014 Mar 26
%  Last Change: 2014 Mar 27
%     Compiler: xelatex
%       Author: gil
%%%%%%%%%%%%%%%%%%%%%%%%%%%%%%%%%%%%%%

%%%%%%%%%%%%%%%%%%%%%%%%%%%%%%%%%%%%%%
\documentclass[12pt,a4paper]{article}
\usepackage{amsmath, amssymb}
\usepackage{fontspec}
\usepackage{graphicx}
\usepackage{booktabs, tabularx}
\usepackage{url}
\usepackage{subcaption}

\begin{document}

\section{The first section}

  \subsection{Subsection}
  \subsubsection{and subsubsection}

\section{Typesetting math}

  \begin{equation}
    Bi = \frac{hr}{k}
    \label{eq:biotnumber}
  \end{equation}

  where $h = 1000 \frac{W}{m^2K}$, $r = .1m$, and $k = 34 \frac{W}{m
  K}$.
  
  This is referencing Equation (\ref{eq:biotnumber}).
  

  \begin{itemize}
    \item Equation labels are prefixed with eq:
    \item Figure labels are prefixed with fig:
    \item Table labels are prefixed with tab:
  \end{itemize}

  \begin{center}

  $x^2$

  $x_2$

  $x^2_2$

  $e^i \pi$

  $e^{i \pi}$

  \end{center}

  \subsection{Equation arrays}

  \begin{align}
    \alpha &= 1 \\
    \beta &= 2 \\ 
    \delta &= 4 
  \end{align}


  \begin{align}
    \gamma &= 23 \\
  \end{align}

  \section{Tables}
    \subsection{Plain tabular} 
\begin{tabular}{lccccccr}
Team              & P & W & D & L & F  & A & Pts \\
\hline
Manchester United & 6 & 4 & 0 & 2 & 10 & 5 & 12  \\
Celtic            & 6 & 3 & 0 & 3 &  8 & 9 &  9  \\
Benfica           & 6 & 2 & 1 & 3 &  7 & 8 &  7  \\
FC Copenhagen     & 6 & 2 & 1 & 3 &  5 & 8 &  7  \\
\end{tabular}

    \vspace{.5cm}

\begin{tabular}{ l | c || r }
  1 & 2 & 3 \\
  4 & 5 & 6 \\
  7 & 8 & 9 \\
\end{tabular}

    \subsection{Floating tables (to add captions and labels)}
    \begin{table}[ht!]
        \centering
        \begin{tabular}{lccccccr}
        Team              & P & W & D & L & F  & A & Pts \\
        \hline
        Manchester United & 6 & 4 & 0 & 2 & 10 & 5 & 12  \\
        Celtic            & 6 & 3 & 0 & 3 &  8 & 9 &  9  \\
        Benfica           & 6 & 2 & 1 & 3 &  7 & 8 &  7  \\
        FC Copenhagen     & 6 & 2 & 1 & 3 &  5 & 8 &  7  \\
        \end{tabular}
      \caption{Manchester United Cheats \label{tab:FCcup}}
    \end{table}

    \clearpage

\section{Figures}
    \subsection{Standard figure}

    \begin{figure}[ht!]
      \centering
      \includegraphics[width=.5\textwidth]{mewbacca.jpg}
      \caption{Mewbacca \label{fig:mewbacca}}
    \end{figure}
   

    \subsection{Multipart Figure}

    %%%Subfloats (multiple figures in a row) requires subcaption package

    \begin{figure}[ht!]
            \centering
            \begin{subfigure}[b]{0.3\textwidth}
                    \includegraphics[width=\textwidth]{mewbacca.jpg}
                    \caption{A mewbacca}
                    \label{fig:mewsub1}
            \end{subfigure}
            ~ 
            \begin{subfigure}[b]{0.3\textwidth}
                    \includegraphics[width=\textwidth]{mewbacca.jpg}
                    \caption{Another mewbacca}
                    \label{fig:mewsub2}
            \end{subfigure}
            ~
            \begin{subfigure}[b]{0.3\textwidth}
                    \includegraphics[width=\textwidth]{mewbacca.jpg}
                    \caption{A third mewbacca!}
                    \label{fig:mewsub3}
            \end{subfigure}
            \caption{Pictures of Mewbacca}\label{fig:threemewbaccas}
    \end{figure}
\end{document}
