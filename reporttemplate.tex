%%%%%%%%%%%%%%%%%%%%%%%%%%%%%%%%%%%%%%
% Comments are started with the % sign
%
%         File: reporttemplate.tex
% Date Created: 2014 Mar 26
%  Last Change: 2014 Apr 02
%     Compiler: xelatex
%       Author: gil
%%%%%%%%%%%%%%%%%%%%%%%%%%%%%%%%%%%%%%

%%%%%%%%%%%%%%%%%%%%%%%%%%%%%%%%%%%%%%
\documentclass[12pt,a4paper]{article}
\usepackage{amsmath, amssymb}
\usepackage{graphicx}
\usepackage{booktabs, tabularx}
\usepackage{url}
\usepackage{subcaption}

\begin{document}

\section{The first section}

  \subsection{Subsection}
  \subsubsection{and subsubsection}

\section{Typesetting math}

  \begin{equation}
    Bi = \frac{hr}{k}
    \label{eq:biotnumber}
  \end{equation}

  where $h = 1000 \frac{W}{m^2K}$, $r = .1m$, and $k = 34 \frac{W}{m
  K}$.
  
  This is referencing Equation (\ref{eq:biotnumber}).
  

  \begin{itemize}
    \item Equation labels are prefixed with eq:
    \item Figure labels are prefixed with fig:
    \item Table labels are prefixed with tab:
  \end{itemize}

  \begin{center}

  $x^2$

  $x_2$

  $x^2_2$

  $e^i \pi$

  $e^{i \pi}$

  \end{center}

  \subsection{Equation arrays}

  \begin{align}
    \alpha &= 1 \\
    \beta &= 2 \\ 
    \delta &= 4 
  \end{align}


  \begin{align}
    \gamma &= 23 \\
  \end{align}

\newpage

  \section{Tables}
    \begin{table}[ht!]
      \centering
        \begin{tabular}{lcccccc}
          AL East & W & L & Pct. & GB & Home & Road \\
          Boston Red Sox & 97 & 65 & 0.599 & — & 53–28 & 44–37\\
          Tampa Bay Rays & 92 & 71 & 0.564 & 5$\frac{1}{2}$ & 51–30 & 41–41\\
          Baltimore Orioles & 85 & 77 & 0.525 & 12 & 46–35 & 39–42\\
          New York Yankees & 85 & 77 & 0.525 & 12 & 46–35 & 39–42\\
          Toronto Blue Jays & 74 & 88 & 0.457 & 23 & 40–41 & 34–47\\
        \end{tabular}
      \caption{AL East Standings 2013 Season\label{tab:aleast2013}}
    \end{table}

    \clearpage

\section{Figures}
    \subsection{Standard figure}

    \begin{figure}[ht!]
      \centering
      \includegraphics[width=.5\textwidth]{mewbacca.jpg}
      \caption{Mewbacca \label{fig:mewbacca}}
    \end{figure}
   

    \subsection{Multipart Figure}

    %%%Subfloats (multiple figures in a row) requires subcaption package

    \begin{figure}[ht!]
            \centering
            \begin{subfigure}[b]{0.3\textwidth}
                    \includegraphics[width=\textwidth]{mewbacca.jpg}
                    \caption{A mewbacca}
                    \label{fig:mewsub1}
            \end{subfigure}
            ~ 
            \begin{subfigure}[b]{0.3\textwidth}
                    \includegraphics[width=\textwidth]{mewbacca.jpg}
                    \caption{Another mewbacca}
                    \label{fig:mewsub2}
            \end{subfigure}
            ~
            \begin{subfigure}[b]{0.3\textwidth}
                    \includegraphics[width=\textwidth]{mewbacca.jpg}
                    \caption{A third mewbacca!}
                    \label{fig:mewsub3}
            \end{subfigure}
            \caption{Pictures of Mewbacca}\label{fig:threemewbaccas}
    \end{figure}
\end{document}
