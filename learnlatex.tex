%%%%%%%%%%%%%%%%%%%%%%%%%%%%%%%%%%%%%%
% Comments are started with the % sign
%
%         File: reporttemplate.tex
% Date Created: 2014 Mar 26
%  Last Change: 2014 Mar 26
%     Compiler: xelatex
%       Author: gil
%%%%%%%%%%%%%%%%%%%%%%%%%%%%%%%%%%%%%%

%%%%%%%%%%%%%%%%%%%%%%%%%%%%%%%%%%%%%%
%The first line of code specifies the document class and the font and
%paper size.  The article document class is a good place to start.  
%Some other options are:
%
%minimal - no fancy bits
%report - for a thesis or a longer report containing chapters
%book - for a book
%letter - nice for cover letters
%
%
%LaTex Libraries extend the functionality of the language.  
%The libraries used here are as follows:
%
%amsmath, amssymb - loading in math symbols and greek letters
%
%fontspec - lets you use fonts available on your system without manually
%installing them
%
%graphicx - for loading in figures/images
%
%booktabs, tabularx - for making pretty tables
%
%There are a lot of libraries available that can vastly extend LaTeX's
%abilities, but these will be sufficient for a solid report.
%
%%%
\documentclass[12pt,a4paper]{article}
\usepackage{amsmath, amssymb}
\usepackage{fontspec}
\usepackage{graphicx}
\usepackage{booktabs, tabularx}

\begin{document}
%Everything inside the \begin{document} and \end{document} tags will be
%compiled into our pdf

Hi!  You should read this document alongside the .tex file that
generated it so that you can get a feel for how the various commands are
rendered in the pdf.  

\section{The first section}
  We can write in plain text here and it will be typeset nicely.  If you
  look at the source code you'll see that the \verb|\section{}| command 
    has automatically made the section title a larger font and also numbered it.  

  \section*{The second section}
    By including an asterisk before the curly-braces, as in
  \verb|\section*{}|,  the automatic numbering is suppressed.  

  \subsection{Subsections}
  \subsubsection{and subsubsections}
  can be created using the \verb|\subsection{}| and
  \verb|\subsubsection{}| commands, respectively.  Again, if you add in
  an asterisk before the curly braces the automatic numbering will be
  suppressed. 

\section{Give up control to the machine}
  Don't worry about where your line breaks are.  \LaTeX\ ignores
  linebreaks in the source text.  
  So
  if
  I
  type
  a
  word
  on
  each
  line
  it
  will
  still
  be
  in
  a
  sentence.

  A full blank line needs to be used to start a new paragraph.  


  Two blank lines will still only make a new paragraph. So stop worrying
  about the formatting and just let \LaTeX\ handle it for you.  

\section{Typesetting math}
  Ok, now we really get started.  \LaTeX\ has two main 'modes'.  Text
  rendering and math rendering.  The default mode is to render text, but
  if we want to write out equations or symbols, we need to tell the
  compiler that we're using math notation.  The first way to do that is
  to insert an equation into a document.
  We have to tell \LaTeX\ that we want an equation environment.  In most
  helper programs there will be a shortcut to insert the appropriate
  lines.  

  \begin{equation}
    Bi = \frac{hr}{k}
    \label{eq:biotnumber}
  \end{equation}

  Notice that an equation number has been automatically inserted.  If we
  have a bunch of equations, we can rearrange them and they'll all be
  renumbered automatically.  In the source, there's also a line that
  reads \verb|\label{eq:biotnumber}| that isn't printed.  Those labels
  can be used to reference the appropriate equation.  If you want to
  draw attention to Equation (\ref{eq:biotnumber}), you can just
  reference the label.  Again, if you reorder your equations, the
  reference will automatically update.  

  Math can also be inserted inline by wrapping the math in \$ signs. So
  you might write: 
  \begin{equation}
    Bi = \frac{hr}{k}
    \label{eq:biotnumber2}
  \end{equation}
  where $h = 1000 \frac{W}{m^2K}$, $r = .1m$, and $k = 34 \frac{W}{m
  K}$.

  \subsection{Subscripts and Superscripts}
  Within a math environment, to type a simple sub- or super-script, you
  can use \_ or \^\ \ ,
  respectively. 

  So to render $x^2$, for instance, the code reads \verb|$x^2$|.  
  Similarly, for $x_2$, you can type \verb|$x_2$|.  And you can combine
  them, so to render $x^2_2$, you simply type \verb|$x^2_2$| or
  \verb|$x_2^2$| -- (\LaTeX\ doesn't care which order they're in)

  If you want more than one character in a sub- or super-script, though,
  you need to wrap them in curly braces.  So we might write out our
  lumped capacitance equation as

  \begin{equation}
    \Theta = e^{-t/\tau}
  \end{equation}

  which in the source is written as \verb|\Theta = e^{-t/\tau}|.
  
  \subsection{Equation arrays}
  Sometimes you need to write out a bunch of equations in a row
  (demonstrating a derivation, for instance).  You can do this with
  consecutive single equation environments, but that can get a bit
  tiresome and might look a little weird.  Instead, we can use the
  \verb|align| environment.  

  \begin{align}
    \alpha &= 1 \\
    \beta &= 2 \\ 
    \delta &= 4 
  \end{align}

  Note that to end a line in an equation array you need to put in a
  double-backslash.  If you add one to the last line, \LaTeX\ will add
  another equation line, but without text it will just be a blank,
  numbered equation.  

  \begin{align}
    \gamma &= 23 \\
  \end{align}


  \section{Inserting Figures or Pictures}
    \begin{figure}[ht]
      \centering
      \includegraphics[width=.5\textwidth]{mewbacca.jpg}
      \caption{Mewbacca \label{fig:mewbacca}}
    \end{figure}

\end{document}
