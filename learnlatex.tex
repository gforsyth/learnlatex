%%%%%%%%%%%%%%%%%%%%%%%%%%%%%%%%%%%%%%
% Comments are started with the % sign
%
%         File: reporttemplate.tex
% Date Created: 2014 Mar 26
%  Last Change: 2014 Apr 02
%     Compiler: xelatex
%       Author: gil
%%%%%%%%%%%%%%%%%%%%%%%%%%%%%%%%%%%%%%

%%%%%%%%%%%%%%%%%%%%%%%%%%%%%%%%%%%%%%
%The first line of code specifies the document class and the font and
%paper size.  The article document class is a good place to start.  
%Some other options are:
%
%minimal - no fancy bits
%report - for a thesis or a longer report containing chapters
%book - for a book
%letter - nice for cover letters
%
%
%LaTex Libraries extend the functionality of the language.  
%The libraries used here are as follows:
%
%amsmath, amssymb - loading in math symbols and greek letters
%
%fontspec - lets you use fonts available on your system without manually
%installing them
%
%graphicx - for loading in figures/images
%
%booktabs, tabularx - for making pretty tables
%
%There are a lot of libraries available that can vastly extend LaTeX's
%abilities, but these will be sufficient for a solid report.
%
%%%
\documentclass[12pt,a4paper]{article}
\usepackage{amsmath, amssymb}
\usepackage{fontspec}
\usepackage{graphicx}
\usepackage{booktabs, tabularx}
\usepackage{url}

\begin{document}
%Everything inside the \begin{document} and \end{document} tags will be
%compiled into our pdf

Hi!  You should read this document alongside the .tex file that
generated it so that you can get a feel for how the various commands are
rendered in the pdf.  

\section{The first section}
  We can write in plain text here and it will be typeset nicely.  If you
  look at the source code you'll see that the \verb|\section{}| command 
    has automatically made the section title a larger font and also numbered it.  

  \section*{The second section}
    By including an asterisk before the curly-braces, as in
  \verb|\section*{}|,  the automatic numbering is suppressed.  

  \subsection{Subsections}
  \subsubsection{and subsubsections}
  can be created using the \verb|\subsection{}| and
  \verb|\subsubsection{}| commands, respectively.  Again, if you add in
  an asterisk before the curly braces the automatic numbering will be
  suppressed. 

\section{Give up control to the machine}
  Don't worry about where your line breaks are.  \LaTeX\ ignores
  linebreaks in the source text.  
  So
  if
  I
  type
  a
  word
  on
  each
  line
  it
  will
  still
  be
  in
  a
  sentence.

  A full blank line needs to be used to start a new paragraph.  


  Two blank lines will still only make a new paragraph. So stop worrying
  about the formatting and just let \LaTeX\ handle it for you.  

  Sometimes, when you absolutely, positively just \textit{HAVE} to have
  a little more whitespace, you can use the \verb|\vspace{}| command,
  which will create a verticle-space of the length you specify.  

  For example, if we want a 2cm space

  \vspace{2cm}
 
  Then we can have a 2cm space.

\section{Typesetting math}
  \LaTeX\ has two main 'modes'.  Text
  rendering and math rendering.  The default mode is to render text, but
  if we want to write out equations or symbols, we need to tell the
  compiler that we're using math notation.  The first way to do that is
  to insert an equation into a document.
  We have to tell \LaTeX\ that we want an equation environment.  In most
  TeX editors there will be a shortcut to insert the appropriate
  lines.  

  \begin{equation}
    Bi = \frac{hr}{k}
    \label{eq:biotnumber}
  \end{equation}

  Notice that an equation number has been automatically inserted. 
%  If we
%  have a bunch of equations, we can rearrange them and they'll all be
%  renumbered automatically.  In the source, there's also a line that
%  reads \verb|\label{eq:biotnumber}| that isn't printed.  Those labels
%  can be used to reference the appropriate equation.  If you want to
%  draw attention to Equation (\ref{eq:biotnumber}), you can just
%  reference the label.  Again, if you reorder your equations, the
%  reference will automatically update.  

  Math can also be inserted inline by wrapping the math in \$ signs. So
  you might write: 

  \begin{equation}
    Bi = \frac{hr}{k}
    \label{eq:biotnumber2}
  \end{equation}

  where $h = 1000 \frac{W}{m^2K}$, $r = .1m$, and $k = 34 \frac{W}{m
  K}$.
  
  \subsection{Referencing Equations and Figures}
  In the equation environment above, there's a line that reads
  \verb|\label{eq:biotnumber}|.  Labels are optional, but if you use
  one, you can refer to the appropriate equation number.  For instance:
  
  \vspace{.5cm} %%This line just adds some whitespace

  This is referencing Equation (\ref{eq:biotnumber}).
  
  \vspace{.5cm} %%This line just adds some whitespace

  If the equations are reordered, the references are automatically
  updated.  There are common conventions for writing labels so that they
  are easier to sort through when you have a large number of them.  

  \begin{itemize}
    \item Equation labels are prefixed with eq:
    \item Figure labels are prefixed with fig:
    \item Table labels are prefixed with tab:
  \end{itemize}

  In many TeX editors, when you begin a reference command (\verb|\ref{|)
  the editor will pop up a list of possible labels to use.

  \subsection{Subscripts and Superscripts}
  Examine the code for this section to see how the following lines are
  written in the code.

  \begin{center}

  $x^2$

  $x_2$

  $x^2_2$

  \end{center}
  
  Note that there's only a single sub- or superscript in each example.
  What if you want to write out multiple exponents?

  \begin{center}
    $e^i \pi$
  \end{center}

  Oops.  If you want more than a single character, you need to wrap the
  entire exponent expression in curly braces.

  \begin{center}
    $e^{i \pi}$
  \end{center}

  \subsection{Equation arrays}
  Sometimes you need to write out a bunch of equations in a row
  (demonstrating a derivation, for instance).  You can do this with
  consecutive single equation environments, but that can get a bit
  tiresome and might look a little weird.  Instead, we can use the
  \verb|align| environment.  

  \begin{align}
    \alpha &= 1 \\
    \beta &= 2 \\ 
    \delta &= 4 
  \end{align}

  Note that to end a line in an equation array you need to put in a
  double-backslash.  If you add one to the last line, \LaTeX\ will add
  another equation line, but without text it will just be a blank,
  numbered equation.  

  \begin{align}
    \gamma &= 23 \\
  \end{align}

  \section{Tables}
    Tables can be a bit of a pain. The following code generates the
    table below.  In the tabular environment, you first specify the
    number of columns with either an l, c or r.  These represent left-,
    center- and right-justified, respectively.  

    An \& is placed to separate individual columns and a row is
    terminated with \verb|\\|.  The \verb|\hline| command creates a
    horizontal line over the width of the table.  
    \begin{verbatim}
    \begin{table}[ht]
      \centering
        \begin{tabular}{lcccccc}
          AL East & W & L & Pct. & GB & Home & Road \\
          Boston Red Sox & 97 & 65 & 0.599 & — & 53–28 & 44–37\\
          Tampa Bay Rays & 92 & 71 & 0.564 & 5$\frac{1}{2}$ & 51–30 & 41–41\\
          Baltimore Orioles & 85 & 77 & 0.525 & 12 & 46–35 & 39–42\\
          New York Yankees & 85 & 77 & 0.525 & 12 & 46–35 & 39–42\\
          Toronto Blue Jays & 74 & 88 & 0.457 & 23 & 40–41 & 34–47\\
        \end{tabular}
      \caption{AL East Standings 2013 Season\label{tab:aleast2013}}
    \end{verbatim}
    \begin{table}[ht]
      \centering
        \begin{tabular}{lcccccc}
          AL East & W & L & Pct. & GB & Home & Road \\
          Boston Red Sox & 97 & 65 & 0.599 & — & 53–28 & 44–37\\
          Tampa Bay Rays & 92 & 71 & 0.564 & 5$\frac{1}{2}$ & 51–30 & 41–41\\
          Baltimore Orioles & 85 & 77 & 0.525 & 12 & 46–35 & 39–42\\
          New York Yankees & 85 & 77 & 0.525 & 12 & 46–35 & 39–42\\
          Toronto Blue Jays & 74 & 88 & 0.457 & 23 & 40–41 & 34–47\\
        \end{tabular}
      \caption{AL East Standings 2013 Season\label{tab:aleast2013}}
    \end{table}

    \vspace{.5cm}
    You can also create horizontal lines to divide up columns by specifying
    either | or || in the column section.  For instance:
    \vspace{.5cm}

    \begin{verbatim}
    \begin{tabular}{ l | c || r }
      1 & 2 & 3 \\
      4 & 5 & 6 \\
      7 & 8 & 9 \\
    \end{tabular}
    \end{verbatim}
    \vspace{.5cm}

    yields
    \vspace{.5cm}

\begin{tabular}{ l | c || r }
  1 & 2 & 3 \\
  4 & 5 & 6 \\
  7 & 8 & 9 \\
\end{tabular}

  \vspace{.5cm}
  There are many helper programs that will automatically generate tables
  from MATLAB or Python or Excel.  You can find a list of them at
  \url{http://en.wikibooks.org/wiki/LaTeX/Tables#Using_spreadsheets}

  \section{Inserting Figures or Pictures}
    The figure on the next page is inserted using the following code
    snippet.  
    
    \begin{verbatim}
    \begin{figure}[H]
      \centering
      \includegraphics[width=.5\textwidth]{mewbacca.jpg}
      \caption{Mewbacca \label{fig:mewbacca}}
    \end{figure}
    \end{verbatim}

    First we begin the figure environment.  The [H] forces the image to
    be placed roughly where the figure call is made.  Without that call,
    Latex will place it wherever there is space for it.  

    The \verb|\centering| command centers the figure (surprise!).  
    The \verb|\includegraphics| line takes the optional argument in the
    square brackets to resize the image.  In this case, it's set to be
    half the width of our text.  

    The filename of the image is in curly braces.  In this example, the
    image is in the same folder as the .tex file.  If you want a
    separate 'images' folder, then you can put 'images/mewbacca.jpg'
    in the curly braces.
    \begin{figure}[ht]
      \centering
      \includegraphics[width=.5\textwidth]{mewbacca.jpg}
      \caption{Mewbacca \label{fig:mewbacca}}
    \end{figure}


\end{document}
